\section{Définition de types}
\label{sec:defining-types}

\subsection{Types personnalisés}

On crée un type avec le mot-clé \hs{data}. Au plus simple, un type réunit plusieurs variables membres:

\begin{haskellcode}
ghci> data MonType = MonType String Integer
ghci> let a = MonType "abc" 123
\end{haskellcode}
(La première occurence de MonType est le nom du type, la seconde est le nom du constructeur)

\subsection{Synonymes de types}

\hs{type} crée un synonyme d'un type existant. Les synonymes et le type auquel ils
renvoient sont interchangeables.

\begin{haskellcode}
type ObjectId = Int16
\end{haskellcode}

Les synonymes créés avec \hs{type} servent principalement à clarifier le sens des champs dans les types utilisateur.

\begin{haskellcode}
type Authors = [String]
type Title = String
type ISBN = Int
type Year = Int
data Book2 = Authors Title Year ISBN
\end{haskellcode}

\subsection{Synonymes «forts»}

\subsubsection{Maybe}

\subsubsection{Either}
