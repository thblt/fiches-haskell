\section{ Modules}
\label{modules}

Haskell dispose d'un mécanisme d'importation de modules.

\subsection{Écrire un module}

Un module a le même nom que le fichier .hs qui le contient, et \qsee{ce nom commence par une majuscule}{identifiers}. La déclaration de module a la syntaxe suivante:

\begin{haskellcode}
-- MyModule.hs
module Mod
    (
      x,
      y,
      z
    ) where
-- code
\end{haskellcode}

Cette déclaration exporte les identifiants x, y et z du code qui la suit. On exporterait la totalité des noms en enlevant la parenthèse, et aucun en la laissant vide.

\todo{Exporter un type sans constructeurs + lien depuis defining-types. Handle est un bon exemple de pourquoi. Aussi RWH 159}

\subsection{Importation de modules}
\label{import}

\begin{hslisting*}
\begin{haskellcode}
-- Commande                          Importé
import Mod                        -- x, y, z, Mod.x, Mod.y, Mod.z
import Mod ()                     -- (Nothing!)
import Mod (x,y)                  -- x, y, Mod.x, Mod.y
import qualified Mod              -- Mod.x, Mod.y, Mod.z
import qualified Mod (x,y)        -- Mod.x, Mod.y
import Mod hiding (x,y)           -- z, Mod.z
import qualified Mod hiding (x,y) -- Mod.z
import Mod as Foo                 -- x, y, z, Foo.x, Foo.y, Foo.z
import Mod as Foo (x,y)           -- x, y, Foo.x, Foo.y
import qualified Mod as Foo       -- Foo.x, Foo.y, Foo.z
import qualified Mod as Foo (x,y) -- Foo.x, Foo.y
\end{haskellcode}
\caption*{D'après \cite{Hudak2000}}
\end{hslisting*}
