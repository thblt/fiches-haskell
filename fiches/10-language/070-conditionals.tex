\section{Structures conditionnelles}
\label{conditionals}

Haskell connaît deux structures conditionnelles: les tests binaires avec \hs{if}, et les cas de \hs{case}.

\subsection{\hs{if ... then ... else}}
\label{if-then-else}

Une clause \hs{if} est une expression, pas une structure de contrôle.
La syntaxe est \hs{if a then b else c}, où \hsFn{a} est une expression de type \hsT{Bool}, \hsFn{b} et \hsFn{c} des expressions d'un type quelconque. Si \hsFn{a} est vraie, l'expression vaut \hsFn{b}, sinon \hsFn{c}.

Comme c'est une expression, on peut affecter son résultat directement à une variable:

\begin{haskellcode}
a = if even x then "pair" else "impair"
\end{haskellcode}

\begin{infobox}
On peut souvent éviter d'employer if en utilisant des \qsee{gardes}{guards}, plus lisibles:

\begin{haskellcode}
compte :: String -> String -> Int -> String
compte sing plur nb = if nb == 0 then "Aucun " ++ sing else if nb == 1 then "Un "++sing else show nb ++ " " ++ plur
\end{haskellcode}

\end{infobox}

\subsection{ \hs{case}}
\label{case}
