\section{Définition de types}
\label{defining-types}

\subsection{Types personnalisés}

On crée un type avec le mot-clé \hs{data}. Au plus simple, un type réunit plusieurs variables membres:

\begin{haskellcode}
ghci> data MonType = MonType String Integer
ghci> let a = MonType "abc" 123
\end{haskellcode}
(La première occurence de MonType est le nom du type, la seconde est le nom du constructeur)

\subsubsection{Accesseurs explicites (\emph{record syntax})}

Un accesseur explicite définit automatiquement une fonction qui prend un objet du  nouveau type en paramètre
et renvoie la valeur

\begin{haskellcode}
data Book = Book {
    bookISBN :: Int,         -- bookISBN :: Book -> Int
    bookAuthors :: [String], -- bookAuthors :: Book -> [String]
    bookTitle :: String      -- bookTitle :: Book -> String
    }
\end{haskellcode}

\subsubsection{Types algébriques}
\label{algebraic-types}

Un type algébrique présente une alternative, en offrant plusieurs constructeurs. Le plus répandu est \hsT{Maybe}:

\begin{haskellcode}
data Maybe a = Nothing | Just a
\end{haskellcode}

\begin{infobox}
\emph{Tous} les types définis avec \hs{data} sont algébriques, même s'ils n'offrent qu'un seul constructeur \parencite{OSullivan2008}.
\end{infobox}

\subsubsection{Types récursifs}

Le type liste natif est défini récursivement. On peut le réimplémenter comme suit:

\begin{haskellcode}
data List a = Empty | Cons a (MyList a)
\end{haskellcode}

\subsection{Synonymes de types}

\hs{type} crée un synonyme d'un type existant. Les synonymes et le type auquel ils
renvoient sont interchangeables.

\begin{haskellcode}
type ObjectId = Int16
\end{haskellcode}

Les synonymes créés avec \hs{type} peuvent servir:

\begin{itemize}
    \item À clarifier le sens des champs dans les types personnalisés sans accesseurs (\hs{type ISBN = Int} pour un type \hsT{Book}, par exemple).
    \item Comme notation abrégée pour des types complexes fréquemment utilisés.
\end{itemize}

\begin{haskellcode}
type Authors = [String]
type Title = String
type ISBN = Int
type Year = Int
data Book2 = Authors Title Year ISBN
\end{haskellcode}

\subsection{Synonymes «forts»}

\todo{cette sous-section}

\todo{$\neq$ data/newtype - question de la performance RWH 157 + conséquences sur les motifs RWH 158}
